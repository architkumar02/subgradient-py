\begin{itemize}
\item Declaring variables: \\
One can declare a variable using the following syntax.
\begin{verbatim}
x = var('x')
\end{verbatim}
The object \verb'x' is a symbolic link to the variable with the identifier string \verb,'x',.
\item Forming expressions: \\
The following line creates an expression named \verb'ex'.
\begin{verbatim}
ex = abs(x - 3) + exp(x)
\end{verbatim}
\verb'SPY' does not attempt to simplify a given expression.
\item Computing values and subgradients of expressions: \\
The following code computes the value and a subgradient at a point given by \verb'varmap'.
\begin{verbatim}
varmap = {'x': 1.5, 'y': -2}
val    = ex.get_value(varmap)
g      = ex.subgrad(varmap)
\end{verbatim}
To represent a point, users should use a dictionary rather than a list of values.
\item Specifying constraints: \\
One can use \verb'leq', \verb'eq', or \verb'geq' to construct a constraint object.
\begin{verbatim}
cons = leq(norm2([x1, x2]), y)
\end{verbatim}
At construction time, \verb'SPY' will automatically check if inequalities and equalities follow the DCP ruleset.
\item Solving an optimization problem:
\begin{verbatim}
myprob = minimize(ex, [cons1, cons2])
(optval, optpoint) = myprob.solve()
\end{verbatim}
The lines above minimizes \verb'ex' subject to two constraints \verb'cons1' and \verb'cons2' using the subgradient method. As a result, the optimal point as well as the objective value at that point will be returned. Advanced users can specify the step size rule used by the method.
\item Available library functions:
\begin{itemize}
\item \verb'abs', \verb'max', \verb'min', \verb'pos'
\item \verb'exp', \verb'log', \verb'log_sum_exp', \verb'rel_entr'
\item \verb'norm1', \verb'norm2', \verb'berhu', \verb'huber'
\item \verb'power', \verb'power_pos', \verb'square', \verb'square_pos', \\ \verb'sqrt', \verb'geo_mean', \verb'quad_over_lin'
\end{itemize}
\end{itemize}
