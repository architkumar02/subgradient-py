\documentclass[12pt]{article}
\usepackage{fullpage,graphicx,psfrag,url}
\usepackage[small,bf]{caption}

\usepackage{amsmath}
\usepackage{amsthm}
\usepackage{amssymb}
\usepackage{verbatim}

\setlength{\captionmargin}{30pt}

\input defs.tex
\newcommand{\sign}{\mathop{\bf sign}}

\bibliographystyle{alpha}

\title{Symbolic Subdifferentiation in Python}
\author{Maurizio Cal\'o and Jaehyun Park\\
EE364B Final Project Progress Report\\
Stanford University, Spring 2010-11}

\begin{document}
\maketitle

\section{Overview}
Motivated by the first two weeks of lectures, we decided to create a
Python package (tentative name: Subgradient-PY, or SPY) that solves
optimization problems using subgradient methods. Using SPY, one will
be able to formulate and solve an optimization problem using the
following syntax (subject to change):
\begin{verbatim}
from spy import *
A = rand(20, 5)
x = vector('x', 5)
prob = spy_prob(minimize, norm(x, 1), [A*x <= 1])
prob.solve()
\end{verbatim}

\section{Implementation}
Along with CVX-like library functions, SPY will implement three
important classes:
\BIT
\item Expression: Any real-valued mathematical expression is an object
of Expression type. It can contain variables whose values are not
predetermined.
\item Constraint: A constraint is an inequality of the form
$(\text{convex}) \le (\text{concave})$ or an equality of the form
$(\text{affine}) = (\text{affine})$.
\item (Optimization) Problem: A problem is a triplet of the form (minimize/maximize, objective, list of constraints).
\EIT

The essential part of the project is to compute subgradients correctly
and efficiently, since all other methods will rely heavily on subgradients.

\section{Current Progress}
\subsection{Expression Class}
Currently, the expression class only allows scalar-valued expressions.

\subsection{Computation of Subgradients}
The recursive computation of subgradients is based on the composition rule

\end{document}
